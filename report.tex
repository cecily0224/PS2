% Options for packages loaded elsewhere
\PassOptionsToPackage{unicode}{hyperref}
\PassOptionsToPackage{hyphens}{url}
%
\documentclass[
]{article}
\usepackage{lmodern}
\usepackage{amssymb,amsmath}
\usepackage{ifxetex,ifluatex}
\ifnum 0\ifxetex 1\fi\ifluatex 1\fi=0 % if pdftex
  \usepackage[T1]{fontenc}
  \usepackage[utf8]{inputenc}
  \usepackage{textcomp} % provide euro and other symbols
\else % if luatex or xetex
  \usepackage{unicode-math}
  \defaultfontfeatures{Scale=MatchLowercase}
  \defaultfontfeatures[\rmfamily]{Ligatures=TeX,Scale=1}
\fi
% Use upquote if available, for straight quotes in verbatim environments
\IfFileExists{upquote.sty}{\usepackage{upquote}}{}
\IfFileExists{microtype.sty}{% use microtype if available
  \usepackage[]{microtype}
  \UseMicrotypeSet[protrusion]{basicmath} % disable protrusion for tt fonts
}{}
\makeatletter
\@ifundefined{KOMAClassName}{% if non-KOMA class
  \IfFileExists{parskip.sty}{%
    \usepackage{parskip}
  }{% else
    \setlength{\parindent}{0pt}
    \setlength{\parskip}{6pt plus 2pt minus 1pt}}
}{% if KOMA class
  \KOMAoptions{parskip=half}}
\makeatother
\usepackage{xcolor}
\IfFileExists{xurl.sty}{\usepackage{xurl}}{} % add URL line breaks if available
\IfFileExists{bookmark.sty}{\usepackage{bookmark}}{\usepackage{hyperref}}
\hypersetup{
  pdftitle={PS2},
  pdfauthor={Cecily Liu},
  hidelinks,
  pdfcreator={LaTeX via pandoc}}
\urlstyle{same} % disable monospaced font for URLs
\usepackage[margin=1in]{geometry}
\usepackage{color}
\usepackage{fancyvrb}
\newcommand{\VerbBar}{|}
\newcommand{\VERB}{\Verb[commandchars=\\\{\}]}
\DefineVerbatimEnvironment{Highlighting}{Verbatim}{commandchars=\\\{\}}
% Add ',fontsize=\small' for more characters per line
\usepackage{framed}
\definecolor{shadecolor}{RGB}{248,248,248}
\newenvironment{Shaded}{\begin{snugshade}}{\end{snugshade}}
\newcommand{\AlertTok}[1]{\textcolor[rgb]{0.94,0.16,0.16}{#1}}
\newcommand{\AnnotationTok}[1]{\textcolor[rgb]{0.56,0.35,0.01}{\textbf{\textit{#1}}}}
\newcommand{\AttributeTok}[1]{\textcolor[rgb]{0.77,0.63,0.00}{#1}}
\newcommand{\BaseNTok}[1]{\textcolor[rgb]{0.00,0.00,0.81}{#1}}
\newcommand{\BuiltInTok}[1]{#1}
\newcommand{\CharTok}[1]{\textcolor[rgb]{0.31,0.60,0.02}{#1}}
\newcommand{\CommentTok}[1]{\textcolor[rgb]{0.56,0.35,0.01}{\textit{#1}}}
\newcommand{\CommentVarTok}[1]{\textcolor[rgb]{0.56,0.35,0.01}{\textbf{\textit{#1}}}}
\newcommand{\ConstantTok}[1]{\textcolor[rgb]{0.00,0.00,0.00}{#1}}
\newcommand{\ControlFlowTok}[1]{\textcolor[rgb]{0.13,0.29,0.53}{\textbf{#1}}}
\newcommand{\DataTypeTok}[1]{\textcolor[rgb]{0.13,0.29,0.53}{#1}}
\newcommand{\DecValTok}[1]{\textcolor[rgb]{0.00,0.00,0.81}{#1}}
\newcommand{\DocumentationTok}[1]{\textcolor[rgb]{0.56,0.35,0.01}{\textbf{\textit{#1}}}}
\newcommand{\ErrorTok}[1]{\textcolor[rgb]{0.64,0.00,0.00}{\textbf{#1}}}
\newcommand{\ExtensionTok}[1]{#1}
\newcommand{\FloatTok}[1]{\textcolor[rgb]{0.00,0.00,0.81}{#1}}
\newcommand{\FunctionTok}[1]{\textcolor[rgb]{0.00,0.00,0.00}{#1}}
\newcommand{\ImportTok}[1]{#1}
\newcommand{\InformationTok}[1]{\textcolor[rgb]{0.56,0.35,0.01}{\textbf{\textit{#1}}}}
\newcommand{\KeywordTok}[1]{\textcolor[rgb]{0.13,0.29,0.53}{\textbf{#1}}}
\newcommand{\NormalTok}[1]{#1}
\newcommand{\OperatorTok}[1]{\textcolor[rgb]{0.81,0.36,0.00}{\textbf{#1}}}
\newcommand{\OtherTok}[1]{\textcolor[rgb]{0.56,0.35,0.01}{#1}}
\newcommand{\PreprocessorTok}[1]{\textcolor[rgb]{0.56,0.35,0.01}{\textit{#1}}}
\newcommand{\RegionMarkerTok}[1]{#1}
\newcommand{\SpecialCharTok}[1]{\textcolor[rgb]{0.00,0.00,0.00}{#1}}
\newcommand{\SpecialStringTok}[1]{\textcolor[rgb]{0.31,0.60,0.02}{#1}}
\newcommand{\StringTok}[1]{\textcolor[rgb]{0.31,0.60,0.02}{#1}}
\newcommand{\VariableTok}[1]{\textcolor[rgb]{0.00,0.00,0.00}{#1}}
\newcommand{\VerbatimStringTok}[1]{\textcolor[rgb]{0.31,0.60,0.02}{#1}}
\newcommand{\WarningTok}[1]{\textcolor[rgb]{0.56,0.35,0.01}{\textbf{\textit{#1}}}}
\usepackage{graphicx,grffile}
\makeatletter
\def\maxwidth{\ifdim\Gin@nat@width>\linewidth\linewidth\else\Gin@nat@width\fi}
\def\maxheight{\ifdim\Gin@nat@height>\textheight\textheight\else\Gin@nat@height\fi}
\makeatother
% Scale images if necessary, so that they will not overflow the page
% margins by default, and it is still possible to overwrite the defaults
% using explicit options in \includegraphics[width, height, ...]{}
\setkeys{Gin}{width=\maxwidth,height=\maxheight,keepaspectratio}
% Set default figure placement to htbp
\makeatletter
\def\fps@figure{htbp}
\makeatother
\setlength{\emergencystretch}{3em} % prevent overfull lines
\providecommand{\tightlist}{%
  \setlength{\itemsep}{0pt}\setlength{\parskip}{0pt}}
\setcounter{secnumdepth}{-\maxdimen} % remove section numbering

\title{PS2}
\author{Cecily Liu}
\date{10/6/2020}

\begin{document}
\maketitle

\hypertarget{r-markdown}{%
\subsection{R Markdown}\label{r-markdown}}

This is an R Markdown document. Markdown is a simple formatting syntax
for authoring HTML, PDF, and MS Word documents. For more details on
using R Markdown see \url{http://rmarkdown.rstudio.com}.

Summary

Introduction

Methodology

In the city of Markham, Ontario, a survey was designed and carried out
where 1000 random participants located in various Walmarts were asked
various questions in relation to their backgrounds and voting
preferences. In each of the 5 locations, 200 participants that answered
`Yes' when asked if they lived in Markham, were chosen for this survey.
Please refer to Appendix 1 for the survey questions.

The sampling approach included a simple random sample where any member
of the 351,552 inhabitants of Markham (as of 2020) all have an equal
chance of being selected, allowing for the sample to represent the
Markham population. The lack of favouritism is represented by the fact
that Walmart is a politically and racially neutral place, where no
matter one's jobs, gender, income, or age, you will be able to see them
there. As well, a stratified random sample by the key characteristics of
interest was chosen. This ensures that the sample will have various
representations of the population.

The surveyers first introduced themselves as volunteers from a Canadian
polling company. After ensuring the participant lives in Markham, they
are asked whether they would like to participate for a 5 dollars Tim
Hortons gift card. Once accepted, they are provided with the survey and
pens. Due to the anonoymous nature of this survey, the participants
information will not be required. When interacting with possible
participants, volunteers need to make sure they are not pressuing them.
Due to the survey size, there will be an approximate cost of 7000
dollars where 5000 dollars will be for the compensation amount, and the
rest for the printing of surveys and food for volunteers. As well, we
estimate that this survey will span 3 days and at least 10 volunteers
are needed.

Taking into consideration the possibility of no response to certain
questions, we will use `NA' to represent these answers and they will be
filtered out. The relatively large sample size will allow our results to
not be too significantly affected. As well, this is useful when
interpreting the data collected.

Survey Results

\#run above and call ``my\_data'' then you can see all the data

\begin{Shaded}
\begin{Highlighting}[]
\KeywordTok{library}\NormalTok{(tibble)}
\KeywordTok{library}\NormalTok{(dplyr)}
\end{Highlighting}
\end{Shaded}

\begin{verbatim}
## 
## Attaching package: 'dplyr'
\end{verbatim}

\begin{verbatim}
## The following objects are masked from 'package:stats':
## 
##     filter, lag
\end{verbatim}

\begin{verbatim}
## The following objects are masked from 'package:base':
## 
##     intersect, setdiff, setequal, union
\end{verbatim}

\begin{Shaded}
\begin{Highlighting}[]
\KeywordTok{library}\NormalTok{(tidyr)}
\KeywordTok{library}\NormalTok{(ggplot2)}

\KeywordTok{set.seed}\NormalTok{(}\DecValTok{1004097443}\NormalTok{)}
\NormalTok{survey_size <-}\StringTok{ }\DecValTok{1000}

\NormalTok{my_data <-}\StringTok{ }\KeywordTok{tibble}\NormalTok{(}
  \DataTypeTok{eligible =} \KeywordTok{sample}\NormalTok{(}\DataTypeTok{x=} \KeywordTok{c}\NormalTok{(}\StringTok{"yes"}\NormalTok{,}\StringTok{"no"}\NormalTok{), }\DataTypeTok{size =}\NormalTok{ survey_size,}
                    \DataTypeTok{replace =} \OtherTok{TRUE}\NormalTok{, }\DataTypeTok{prob =} \KeywordTok{c}\NormalTok{(}\FloatTok{0.9}\NormalTok{,}\FloatTok{0.1}\NormalTok{)),}
  \DataTypeTok{Race =} \KeywordTok{sample}\NormalTok{(}\DataTypeTok{x=}\KeywordTok{c}\NormalTok{(}\StringTok{"Asian"}\NormalTok{, }\StringTok{"Africa Americans"}\NormalTok{, }\StringTok{"Christian"}\NormalTok{), }\DataTypeTok{size =}\NormalTok{ survey_size, }\DataTypeTok{replace =} \OtherTok{TRUE}\NormalTok{, }\DataTypeTok{prob=}\KeywordTok{c}\NormalTok{(}\FloatTok{0.45}\NormalTok{,}\FloatTok{0.15}\NormalTok{,}\FloatTok{0.4}\NormalTok{)),}
  \DataTypeTok{gender =} \KeywordTok{sample}\NormalTok{(}\DataTypeTok{x =} \KeywordTok{c}\NormalTok{(}\StringTok{"male"}\NormalTok{,}\StringTok{"female"}\NormalTok{), }\DataTypeTok{size =}\NormalTok{ survey_size,}
               \DataTypeTok{replace =} \OtherTok{TRUE}\NormalTok{, }\DataTypeTok{prob =} \KeywordTok{c}\NormalTok{(}\FloatTok{0.45}\NormalTok{,}\FloatTok{0.55}\NormalTok{)),}
  \DataTypeTok{education =} \KeywordTok{sample}\NormalTok{(}\DataTypeTok{x=} \KeywordTok{c}\NormalTok{(}\StringTok{"high school not completed"}\NormalTok{,}\StringTok{"high school completed"}\NormalTok{,}\StringTok{"college/university complete"}\NormalTok{,}
                          \StringTok{"master completed"}\NormalTok{,}\StringTok{"phd or higher completed"}\NormalTok{), }\DataTypeTok{size =}\NormalTok{ survey_size,}\DataTypeTok{replace =} \OtherTok{TRUE}\NormalTok{, }\DataTypeTok{prob =} \KeywordTok{c}\NormalTok{(}\FloatTok{0.25}\NormalTok{,}\FloatTok{0.3}\NormalTok{,}\FloatTok{0.3}\NormalTok{,}\FloatTok{0.1}\NormalTok{,}\FloatTok{0.05}\NormalTok{)),}
  \DataTypeTok{income =} \KeywordTok{sample}\NormalTok{(}\DataTypeTok{x =} \KeywordTok{c}\NormalTok{(}\StringTok{"0-40000"}\NormalTok{,}\StringTok{"40000-80000"}\NormalTok{,}\StringTok{"80000 or more"}\NormalTok{), }\DataTypeTok{size =}\NormalTok{ survey_size, }\DataTypeTok{replace =} \OtherTok{TRUE}\NormalTok{, }\DataTypeTok{prob =} \KeywordTok{c}\NormalTok{(}\FloatTok{0.35}\NormalTok{,}\FloatTok{0.45}\NormalTok{,}\FloatTok{0.2}\NormalTok{)),}
  \DataTypeTok{Satisfaction =} \KeywordTok{sample}\NormalTok{(}\DataTypeTok{x=}\DecValTok{1}\OperatorTok{:}\DecValTok{5}\NormalTok{, }\DataTypeTok{size =}\NormalTok{ survey_size, }\DataTypeTok{replace =} \OtherTok{TRUE}\NormalTok{, }\DataTypeTok{prob=}\KeywordTok{c}\NormalTok{(}\FloatTok{0.1}\NormalTok{,}\FloatTok{0.3}\NormalTok{,}\FloatTok{0.3}\NormalTok{,}\FloatTok{0.2}\NormalTok{,}\FloatTok{0.1}\NormalTok{)),}
  \DataTypeTok{VoteForParty =} \KeywordTok{sample}\NormalTok{(}\DataTypeTok{x=}\KeywordTok{c}\NormalTok{(}\StringTok{"YES"}\NormalTok{,}\StringTok{"NO"}\NormalTok{), }\DataTypeTok{size =}\NormalTok{ survey_size, }\DataTypeTok{replace =} \OtherTok{TRUE}\NormalTok{,}\DataTypeTok{prob=}\KeywordTok{c}\NormalTok{(}\FloatTok{0.5}\NormalTok{,}\FloatTok{0.5}\NormalTok{)),}
  
\NormalTok{)}
\KeywordTok{head}\NormalTok{(my_data)}
\end{Highlighting}
\end{Shaded}

\begin{verbatim}
## # A tibble: 6 x 7
##   eligible Race       gender education        income   Satisfaction VoteForParty
##   <chr>    <chr>      <chr>  <chr>            <chr>           <int> <chr>       
## 1 yes      Christian  male   high school not~ 40000-8~            5 NO          
## 2 yes      Africa Am~ female high school not~ 40000-8~            2 YES         
## 3 yes      Asian      female master completed 40000-8~            3 NO          
## 4 yes      Christian  female high school not~ 80000 o~            3 YES         
## 5 yes      Africa Am~ male   college/univers~ 0-40000             2 YES         
## 6 yes      Africa Am~ female high school not~ 80000 o~            3 NO
\end{verbatim}

\begin{Shaded}
\begin{Highlighting}[]
\NormalTok{my_data_yes<-}
\StringTok{  }\NormalTok{my_data}\OperatorTok
\StringTok{  }\KeywordTok{filter}\NormalTok{(my_data}\OperatorTok{$}\NormalTok{eligible }\OperatorTok{==}\StringTok{ "yes"}\NormalTok{)}

\NormalTok{my_data_yes}
\end{Highlighting}
\end{Shaded}

\begin{verbatim}
## # A tibble: 897 x 7
##    eligible Race      gender education        income   Satisfaction VoteForParty
##    <chr>    <chr>     <chr>  <chr>            <chr>           <int> <chr>       
##  1 yes      Christian male   high school not~ 40000-8~            5 NO          
##  2 yes      Africa A~ female high school not~ 40000-8~            2 YES         
##  3 yes      Asian     female master completed 40000-8~            3 NO          
##  4 yes      Christian female high school not~ 80000 o~            3 YES         
##  5 yes      Africa A~ male   college/univers~ 0-40000             2 YES         
##  6 yes      Africa A~ female high school not~ 80000 o~            3 NO          
##  7 yes      Christian male   college/univers~ 0-40000             4 NO          
##  8 yes      Asian     female college/univers~ 0-40000             2 NO          
##  9 yes      Christian male   college/univers~ 0-40000             4 YES         
## 10 yes      Christian female high school not~ 80000 o~            3 NO          
## # ... with 887 more rows
\end{verbatim}

Note: 1.Select (race, vote\_for\_conservative), we can conclude that
Asian people are tended to vote for the conservative while as other
races tend to not support the conservative. 2.select (education,
satisfaction): higher education will result higher satisfaction for the
conservative. 3. select(income,vote\_for\_conservative): higher income
will result lower vote for the conservative because they would like to
vote for the liberal party because the liberal are for the rich. 4. the
more satisfaction of people, they are going to vote for our party.

Discussion

Appendix

\begin{enumerate}
\def\labelenumi{\arabic{enumi}.}
\tightlist
\item
  survey
\end{enumerate}

References

\end{document}
